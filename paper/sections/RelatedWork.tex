\section{Related Work}\label{sec:related}

%To the best of our knowledge, our work proposes the first solution 
%working entirely on a mainstream OO languages, that deals with the 
%problem boilerplate code in traversals. 

This section discusses related work.

\paragraph{Adaptive Object-Oriented Programming (AOOP)}
AOOP~\cite{DemeterBook} is a programming style where it is possible to
select parts of a structure that should be visited. This is useful to
do traversals on complex structures and focus only on the interesting
parts of the structure relevant for computing the final output. The
original approach to AOOP was based on a domain-specific
language~\cite{DemeterBook}. DJ is an implementation of AOOP in Java
using reflection~\cite{DJ}. More recently, inspired by AOOP,
DemeterF~\cite{OOGP} improved on previous approaches by providing
support for safe traversals, generics and data-generic function
generation. Compared to \name most AOOP approaches are not
type-safe. Only in DemeterF a custom type system was designed to
ensure type-safety of generic functions. However, in contrast to
\Name, DemeterF requires a new language and it is unclear how DemeterF
traversals interact with extensibility.

\paragraph{Structure-Shy Traversals with Visitors}
Visser~\cite{visser01visitor} provided some visitor
combinators that can express interesting traversal strategies in the
{\sc visitor} pattern~\cite{gof}. We applies some similar idea like identity
transformation in simple transformation, but our work targets at
traversal control in Object Algebras.\bruno{Tijs}\bruno{More work: 
The essence of the visitor pattern, and follow-up work to improve 
performance.}

\paragraph{Strategic Programming}\bruno{Tijs}

\paragraph{Object Algebras.} \name traversals are based on
object algebras~\cite{bruno12oa}. The original motivation for object
algebras was as a design pattern for OO programming that allowed
improved extensibility and modularity of programs.  Using object
algebras it is possible to solve the well-known ``Expression
Problem''~\cite{wadler98expression-problem}.  Later
work~\cite{oliveira13fop,rendel14attributes} has explored the use of
\emph{object algebra combinators}, and generalizations of object
algebras to improve expressiveness and modularity. In particular it
has been claimed that object algebras can be used to do
\emph{feature-oriented programming}~\cite{oliveira13fop}, and to
encode \emph{attribute grammars}~\cite{rendel14attributes}. One domain
where object algebras are especially useful is in the implementation of
(extensible) languages.  The QL language used in our case study is
based on Gouseti et al.~\cite{gouseti14extensible} work. That work
provides a realistic implementation of an extensible
\emph{domain-specific language} using object algebras. In contrast to
our work, previous work on object algebras was mostly motivated by
improved programming support for extensibility and
modularity. Our work shows that object algebras are also useful 
to solve a different problem: how to traverse complex structures 
without boilerplate code. The combination of extensibility and 
structure-shy type-safe traversals also adds a new dimention to 
our work that, as far as we know, as not been explored previously.


%However, such work has not addressed the problem of how to
%eliminate boilerplate in traversals. Our work shows that the
%combination of object algebras representing basic forms of traversals
%(queries and transformations) and standard OO inheritance provides a
%solution to the traversal boilerplate problem.

\textit{Structure-Shy Traversals in Functional Programming.} 
In the functional programming community like Haskell, much research
has been done on traversal control of large structures. Lammel and
Peyton Jones' ``Scrap your Boilerplate''~\cite{ralf03syb,?,?} series
introduced a practical design pattern for doing type-safe
structure-shy traversals in tree structures, and were a source 
of inspiration. Quite different implementation techniques ... 
SyB traversals are notoriously slow due to the use of some run-time 
reflection techniques.
%The notions of queries 
%and transformations are inspired by their work.
Bringert\cite{bjorn08acf} introduced useful compositional
functions to help construct final results in
Haskell. Lammel\cite{ralf00banana} proposed a polytypic programming
approach for generalized and basic folds. These fold algebras scale up
applications involving large systems of mutually recursive
datatypes. These works all try to optimize traversal control of large
structures in functional programming paradigm, while our work solves a
similar problem in Object Algebras, a programming style in Object
Oriented Programming paradigm.

In summary, prior to our work, research has been done on object
algebras and composition problem of this programming style. In the
functional programming world and with visitor pattern, traversal
control in large structures is also explored. Different from these
work, we explored techniques helping write generic queries and
transformations traversing large tree structures with Object Algebras.

