\section{Case Study}

We have used the Object Algebras framework in the implementation of a simple DSL for Questionnaires, called QL~\cite{GPCE}.
A questionnaire is rendered as an interactive form where, depending on user actions, new questions may appear, or values may be computed. 
QL programs consist of lists of labeled, typed questions.
Questions can be answerable, meaning the user has to enter some data, or computed.
In the latter case the question is defined by an expression.
A conditional if-then-else construct allows questions to visually appear only when a certain condition is true. 

Apart from the parser, the QL implementation consists of three main components.
First, a type checker is defined to check that expressions are used consistently and that no cyclic dependencies between questions exist.
Second, the  interpreter renders questionnaires as a GUI and ensures and computed questions are evaluated.
Finally, a pretty printer allows serialization of questionnaires to text. 

A recent experiment explored the implementation of QL using Object Algebras~\cite{GPCE14}.
This involved a lot of repetitive boilerplate code in the implementation.
We have used \name to remove such boilerplate for the following query operations:
\begin{itemize}
\item Collect type environment: construct a mapping from question name to question type. This mapping is used during type checking.
  \item Reference graph: construct a graph where nodes represent name occurences, and edge represent the use-define relation.
\item Collect question dependencies: extract a binary relation capturing the data and control dependencies between questions. This is used in detecting cyclic dependencies.
\end{itemize}


\begin{table}[t]
  \centering
  \begin{tabular}{l|r|r|r}
    Operation        & Original SLOC & \name SLOC & Reduction (\%) \\\hline
    Type Environment &               &            &                \\
    Reference graph  &               &            &                \\
    Dependencies     &               &            &                \\\hline
    \textit{Total}   &               &            &                \\
  \end{tabular}
  \caption{Effect of using \name on the implementation of QL\label{TBL:qlresults}}
\end{table}



\subsection{Desugaring Language Extensions}



\subsection{\name performance vs Vanilla ASTs}

1 Plot/table

