\section{Desugaring Transformations}\label{sec:heterotrans}

\hl{A new section.}

In Section~\ref{sec:transformations}, we presented transformations, as well as the generic traversals generated by \name. Although different constructors can be used in transforming a data structure, the generic transformations generated by \name are actually type-preserving. We can observe from a transformation algebra that the base algebra always has the same type as the structure to be transformed.

On the other hand, transformations between different types are also useful under some circumstances. For instance, the desugaring in developing a compiler usually involves two ASTs, and they are only similar in some constructors. In that case, transformations return a different type of the structure.

As an example, we still use the interface \lstinline{ExpAlg}. Consider a new constructor called \lstinline{Double}, which multiplies an expression by two. We present it as an Object Algebra interface:

\lstinputlisting[linerange=6-9]{../ObjectAlgebras/src/trees/DoubleAlg.java} % APPLY:linerange=DOUBLE_TREE

\noindent Note that one can also define \lstinline{DoubleAlg} with four constructors, including those three in \lstinline{ExpAlg}. But this implementation is more concise with \name framework.

An expression with type \lstinline{ExpAlg<Exp> & DoubleAlg<Exp>} can be created using four constructors, where a transformation can for example be desugaring the case \lstinline{Double(e)} into \lstinline{Add(e, e)}. The other cases just remain unchanged. Below is an implementation by hand:

\lstinputlisting[linerange=7-11]{../ObjectAlgebras/src/example_DoubleAlg1/Desugar.java} % APPLY:linerange=DESUGAR_DOUBLE

\noindent With the \name framework one can easily implement \lstinline{Desugar} as follows: \hl{(add generated code to Appendix?)}

\lstinputlisting[linerange=7-11]{../ObjectAlgebras/src/example_DoubleAlg2/Desugar.java} % APPLY:linerange=DESUGAR_DOUBLE_SHY

\noindent Now one can create a structure of an expression with desugaring applied as follows:

\lstinputlisting[linerange=8-13]{../ObjectAlgebras/src/example_DoubleAlg2/Test.java} % APPLY:linerange=MAKE_DOUBLE_EXP

\noindent Here an expression ``\lstinline{5 + 2a}'' is created before transformation is applied. The base algebra \lstinline{alg} can for example be a pretty printer. By invoking this \lstinline{makeExp} method, a pretty printer can now return the result \lstinline{(5 + (a + a))}.
